\chapter{Techniques}

\section{Scanning Tunneling Microscopy - STM}

The main technique used in this study is STM, and more specifically the Aarhus STM. The working principle of the STM is the tunneling transmittivity which depends exponentially on the distance between a sharp tip and the sample. A bias is applied between the sample and the tip, and at relatively large distances the barrier that the electrons perceive is too high, and no tunneling occurs. As the distance between the sample and the tip is reduced the probability of an electron tunneling through the vacuum barrier increases until a point where tunneling happens, and the current can be measured. The tunneling current is reduced by about a factor of 10 for every ångstrøm.\cite{STMbinnig} This ensures that the STM is a very precise technique and hence essential for surface scientists. In order to achieve the tunneling current it is necessary that the tip as well as the sample is conducting.\\
In order to create an image, the surface of the sample is scanned with the tip, which is moved by piezo-elements. The tunneling current is dependant on distance between tip and sample as well as the local density of states (LDOS). A map of the LDOS is therefore obtained when the tip is scanned across the surface of the sample. This is used to indirectly create an image of the surface.\\


\section{Temperature Programmed Desorption - TPD}


\section{Low Energy Electron Diffraction - LEED}
