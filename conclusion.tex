\chapter{Conclusion and future perspectives}

\section{conclusion and future perspectives}
Hydrogenation of graphene on Ir(111) by using vibrationally excited molecules has been investigated throughout this project. Pure graphene on Ir(111) was initially investigated along with the defects by carbon vacancies. The moiré pattern arising from graphene on Ir(111) has been observed by using STM. Hydrogenation by vibrationally excited molecules was proven successful and the influence of the flux of atomic hydrogen has been investigated. This was done by scanning the hydrogenated surface following doses with the atomic beam source at temperatures of 1745\degree C, 1593 \degree C, and 1343\degree C. Having a D$_2$ chamber pressure of $5 \cdot 10^{-7}$mbar during dosing, it was found that hydrogen coverages of after 20 min doses were; 35\%, 28\%, and a few \%.\\
TPD measurements revealed that hydrogen dimers only form after hydrogenation by atomic hydrogen. Furthermore it was calculated from the TPD measurements that hydrogen on graphene on Ir(111) in the HCP and FCC sites has an energy of desorption around 2.0-2.1$\pm$0.41eV. \\
\\
Her mangler stadig noget. Blandt andet hvilke eksperimenter der ville være relevante hvis der var mere tid.
