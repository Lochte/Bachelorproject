\chapter{Introduction}

With an ever rising demand of energy, the world is in need of great energy storage devices. Through the ages a wide range of these devices have been invented including electrochemical devices such as Li-ion batteries, electrical capacitors or mechanical flywheels. These are only some devices capable of storing energy. However, with a steadily increasing global average temperature, a cleaner and better way of storing energy is demanded. Research within hydrogen storage is therefore massive, since water is the only product from the combustion of hydrogen. Hydrogen has proven itself difficult to store at high hydrogen densities. Liquid hydrogen requires cryogenic tanks at a temperature around 20K. Using materials to store hydrogen instead is a promising solution to the hydrogen storage problem. One of such materials is graphene. Since graphene is one atom thick, its surface to volume ratio is very high. Therefore a high weight percent of stored hydrogen is obtainable. Furthermore the release of hydrogen from graphene can be controlled, and graphene does therefore show promising prospect.\\
This bachelor project aims at describing the hydrogenation of graphene using vibrationally excited molecules. A graphene on Ir(111) surface is chosen as the substrate. The surface is investigated by STM and a qualitative description of the hydrogenation is explained with theoretical backing. Furthermore the desorption of hydrogen from the surface is investigated by temperature programmed desorption.


All of the experiments conducted in this bachelors thesis is done in the Surface Dynamics Lab during the first six months of 2016. I am grateful to Liv Hornekær for being a wonderful supervisor and giving me the opportunity to work in the lab. Furthermore Line Kyhl has been indispensable with guidance of using 'The Coal Chamber', as well as being supportive during the entire procedure. A big thanks to everyone in the Surface Dynamics Lab including; Richard Balog, Andrew Cassidy, Anders Lind Skov, Frederik Doktor, Susanne Halkjær, Uffe Holm and Kasper Medum Rasmussen. All of which have been helpful, kind and great company during this project.\\
A special thanks goes to Antonija Grubisic Cabo for conducting the LEED experiments. The LEED setup will not be discussed in this report.
