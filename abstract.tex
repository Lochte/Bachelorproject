\chapterstyle{abstract}
\vspace{6cm}
\chapter*{Abstract}
\chapterstyle{box}

The hydrogenation of graphene on Ir(111) using vibrationally excited molecules has been studied. Vibrationally excited D$_2$ molecules with an energy exceeding the dissociation energy of D$_2$ on graphene are produced by associative recombination of atomic hydrogen on chamber walls. This implies that vibrational levels of at least v''=7 are achieved. Atomic hydrogen was produced using a hydrogen atom beam source (HABS), operating at temperatures of 1343\degree C, 1593\degree C and 1745\degree C. Having a D$_2$ chamber pressure of $5 \cdot 10^{-7}$ mbar during dosing, it was found that hydrogen coverages on graphene after 20 min D$_2$ exposure were; 35\%, 28\%, and a few \% with HABS temperatures of 1745\degree C, 1593\degree C and 1343\degree C, respectively. Furthermore, the graphene/Ir(111) surface was saturated after 60 min exposure to excited molecules at 1745\degree C.\\
The hydrogenated graphene/Ir(111) surface is examined by STM, and the observations show that hydrogen adsorption primarily happens at the FCC site of the moire unit cell. The periodicity of the hydrogenated surface is equal to the moiré periodicity of clean gaphene on Ir(111).\\
In order to investigate the dissociation of D$_2$ further, temperature programmed desorption (TPD) measurements were carried out. These show that the desorption of hydrogen follows a first order desorption, with desorption energies of $ 2.1 \pm 0.4$ eV. Following exposure to excited molecules it is found that formation of hydrogen dimers is absent on the graphene on Ir(111) surface. However, dimer peaks appear following dosage of atomic hydrogen.\\
Bilayered graphene patches were observed by STM. Images reveal that the hydrogenation of the bilayer patch is absent, due to its free standing graphene resemblance. Furthermore it is observed that a new kind of moiré pattern is present at bilayer patches with graphene sheets presumably rotated compared to each other.
